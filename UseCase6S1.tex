\documentclass[letterpaper]{article}
\usepackage{latexsym}
\usepackage{amsbsy}
\usepackage{graphicx}
\begin{document}
\noindent
\textbf{Use Case S1:  Archive Weather Data}\\
\textbf{Scope: }Weather Station\\
\textbf{Level:  }System Goal\\
\textbf{Primary Actor:  }System\\
\textbf{Scenario:  }Archive Temperature Data\\
\textbf{Related Use Cases:  }Use Case SA1:  \emph{The System 
Administrator shall set the measurement time for all Weather Data
for the System.}\\
\textbf{Stakeholders and Interests:  }
\begin{itemize}
\item Users:  who want to view archived weather data.
\item Weather Bureaus:  that want to view archived data from other
geographic locations. 
\end{itemize}
\textbf{Preconditions:  }The Weather Station is running.  The
Archiving System running and able to receieve data for archiving.\\
\textbf{Success Guarantees (Postconditions):  }Data is sent to the
archiving system.  The no error indication is returned from the
archiving system.\\\\
\textbf{Main Success Scenario:  }\\\\
\begin{tabular}{|p{6cm}|p{6cm}|}\hline
\textbf{System} & \textbf{Archiving System}\\\hline
1. Send data to be archived & \\\hline
2. Send type of data to be archived & \\\hline
3. Send the time, day, date of the measured data & \\\hline
& 4.  Archives the data in the appropriate location (Temperature,
Humidity, Barometer, etc...)\\\hline
\end{tabular}\\\\
\textbf{Alternative Flows:  }\\
4a.  If the Archiving System is not available:\\
4a1.  The System saves the data in an ``alternative format",
regardless.\\\\
4b.  If Archiving System has no system for the archiving the weather
data:\\
4b1.  The Archiving System will create the system for archiving the
weather data.\\
4b2.  The Archiving System will then save the data requested for
archiving.\\\\
4c.  If the Archiving System cannot save the data:\\
4c1.  The Archiving System will alert the System.\\
4c2.  The Archiving System will attempt to save the data again.\\
4c3.  The System saves the data in an ``alternative format",
regardless.
\end{document}
